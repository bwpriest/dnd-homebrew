\documentclass[10pt,twoside,twocolumn]{article}
\usepackage[bg-print]{dnd} % Options: bg-a4, bg-letter, bg-full, bg-print, bg-none.
\usepackage[ngerman]{babel} % Trennungsregeln und autom. Überschriften in n. dt. RS
\usepackage[utf8]{inputenc}
\usepackage[T1]{fontenc}
\usepackage{array}

\newlength\Origarrayrulewidth
\newcommand{\Cline}[1]{%
  \noalign{\global\setlength\Origarrayrulewidth{\arrayrulewidth}}%
  \noalign{\global\setlength\arrayrulewidth{2pt}}\cline{#1}%
  \noalign{\global\setlength\arrayrulewidth{\Origarrayrulewidth}}%
}

\newcommand\Thickvrule[1]{%
  \multicolumn{1}{!{\vrule width 2pt}c!{\vrule width 2pt}}{#1}%
}

\newcolumntype{L}[1]{>{\raggedright\let\newline\\\arraybackslash\hspace{0pt}}m{#1}}
\newcolumntype{C}[1]{>{\centering\let\newline\\\arraybackslash\hspace{0pt}}m{#1}}
\newcolumntype{R}[1]{>{\raggedleft\let\newline\\\arraybackslash\hspace{0pt}}m{#1}}


% Start document
\begin{document}
\fontfamily{ppl}\selectfont % Set text font

% Your content goes here

%\section{Main Section}
%\lipsum[1] % filler text
%
%\subsection{Fun with boxes}
%\subsubsection{Even more fun!}
%
%\begin{commentbox}{Neat Green Box!}
%	\lipsum[1]
%\end{commentbox}
%
%\begin{quotebox}
%	As you approach this template you get a sense that the blood and tears of many generations went into its making. A warm feeling welcomes you as you type your first words.
%\end{quotebox}
%
%\newpage % Acts as columbreak because of twocolumn option; for pagebreak use \clearpage
%
%% For more columns, you can say \begin{dndtable}[your options here}.
%% For instance, if you wanted three columns, you could say
%% \begin{dndtable}{XXX}. The usual host of tabular parameters are
%% aailable as well.
%\header{Nice table}
%\begin{dndtable}
%   	\textbf{Table head}  & \textbf{Table head} \\
%   	Some value  & Some value \\
%   	Some value  & Some value \\
%   	Some value  & Some value
%\end{dndtable}
%
%\begin{paperbox}{Do the Players need direction?}
%	\lipsum[1]
%\end{paperbox}
%
%% You can optionally not include the background by saying
%% begin{monsterboxnobg}
%\begin{monsterbox}{Monster Foo}
%	\textit{Small metasyntactic variable (goblinoid), neutral evil}\\
%	\hline
%	\basics[%
%	armorclass = 12,
%	hitpoints  = 16 (3d8 + 3),
%	speed      = 50 ft
%	]
%	\hline
%	\stats[
%    STR = \stat{12}, % This stat command will autocomplete the modifier for you
%    DEX = \stat{7}
%	]
%	\hline
%	\details[%
%	% If you want to use commas in these sections, enclose the
%	% description in braces.
%	% I'm so sorry.
%	languages = {Common Lisp, Erlang},
%	]
%	\hline \\[1mm]
%	\begin{monsteraction}[Monster-super-powers]
%		This Monster has some serious superpowers!
%	\end{monsteraction}
%	\monstersection{Actions}
%	\begin{monsteraction}[Generate text]
%		This one can generate tremendous amounts of text! Though only when it wants to.
%	\end{monsteraction}
%
%	\begin{monsteraction}[More actions]
%    See, here he goes again! Yet more text.
%	\end{monsteraction}
%\end{monsterbox}

\section{Wands \& Feelings}

    \begin{charactersheet}{My name is... \underline{\hspace{4cm}}}
    	You are a new student at a wizarding school of great renown.
%	Some outstanding students excel in  athleticism and grit, others knowledge and intellect, and others still passion and eloquence. 
%	Many are a mixture. 
%	You may distribute a total of three points between {\bf body}, {\bf mind}, and {\bf heart}

	\charactersection{Moves}
		When attempting to cast a spell, perform a feat of athleticism, or influence someone, the gm may call for a roll.
		This is called a 'move.'
		You will roll 2d6 and add your {\bf body}, {\bf mind}, or {\bf heart} modifier, depending on what you are trying to do.
		The GM may apply additional modifiers.
		The result of the roll tells us what happens.
		
		\begin{itemize}
			\item {\bf On a 10+}: Your move succeeds. You narrate the effects of your move and the GM fills in the blanks.
			\item {\bf On a 7-9}: Your move succeeds, but the GM will provide you with a disadvantage or ugly choice as a consequence.
			\item {\bf On a 6-}: Your move fails. The GM narrates the consequences.
		\end{itemize}

%        \charactercombatstats[armorclass={12},
%                             initiative={+1},
%                             speed={25},
%                             maxhitpoints=19,
%                             hitpoints=16,
%                             hitdice={2d8}
%                             ]


	\charactersection{Spellcasting}
		When you use a spellcasting move, you describe the spell you are attempting to perform. 
		You may use a spell that exists in the lore, or you may describe a new spell.
		You decide whether to cast your spell for {\bf precision} or {\bf power}.
		You and the GM decide what these adjectives mean in context.
		\begin{itemize}
			\item {\bf Precision}: Your magic is elegant and accurate.
			\item {\bf Power}: You magic is forceful and imprecise.
		\end{itemize}
	
	\charactersection{Attributes}
		You have 3 points to distribute amongst your {\bf Body}, {\bf Mind}, and {\bf Heart} scores.

            \begin{tabular}{c | C{3cm} | L{5cm}}
            \Cline{2-2}
            {\large \bf Body} & \Thickvrule{ \,\,\,\,\,\,\,\,\,\,\,\,\,\,\,\,} & 	Your physical fitness. 
            													Moves that call for {\bf body} 
    													rolls include feats of strength, 
    													skillful broom flying, and dodging danger.\\
    	\Cline{2-2}
            {\large \bf Mind} & \Thickvrule{ \,\,\,\,\,\,\,\,\,\,\,\,} & 	Your intellect. Moves that call 
            													for {\bf mind} rolls include casting for 
    													{\bf precision} and recalling magical lore\\
    	\Cline{2-2}
            {\large \bf Heart} & \Thickvrule{ \,\,\,\,\,\,\,\,\,\,\,\,} & 	Your passion and charisma. 
            													Moves that call for {\bf heart} 
    													rolls include casting for {\bf power} 
    													and persuading others.\\
    	\Cline{1-3}
            {\large \bf Health} & \Thickvrule{\,\,\,\,\,\,\,\,\,\,\,\,} & 	Your sturdiness. Set max equal to 3+{\bf Body}. 
            										If your health is reduced to zero, you fall
											unconscious.\\
    	\Cline{2-2}
            \end{tabular}

%        \stats[STR=\stat{10}, 
%               DEX=\stat{12}, 
%               CON=\stat{10}, 
%               INT=\stat{14}, 
%               WIS=\stat{13}, 
%               CHA=\stat{16}
%               ]

%        \hline

        \charactersection{Wand}
            Every wizard must have a wand.
            Your choice of wand gives you an advantage that you can invoke once per session.
            \begin{itemize}
                \item[o] {\bf Dragon Heartstring}: You automatically roll 10 on a spellcasting roll at the cost of 1 health.
                \item[o] {\bf Phoenix Feather}: You cast a spell for {\bf precision} \emph{and} {\bf power}. Use either {\bf Mind} or {\bf Heart}.
                \item[o] {\bf Unicorn Hair}: You change a spellcasting roll of 6- into a 7.
            \end{itemize}

        \charactersection{House}
		Every wizard must belong to a house.
		Your choice of house gives you an advantage that you can invoke once per session.
            \begin{itemize}
                \item[o] {\bf Griffindor}: When you would take damage, instead take no damage.
                \item[o] {\bf Hufflepuff}: After an ally makes a move, apply +2 to their roll.
                \item[o] {\bf Ravenclaw}: You automatically roll a 10 when attempting to recall lore.
                \item[o] {\bf Slytherin}: You ignore any negative consequences of one of your moves.
            \end{itemize}

        \charactersection{Favorite Subject}
		All wizards have different talents.
		You will be slightly better (+1 to roll) at performing spells or recalling lore related to your favorite subject.
            \begin{itemize}
                \item[o] {\bf Charms}: manipulating objects
                \item[o] {\bf DADA}: offensive/defensive spells
                \item[o] {\bf Potions}: brewing magical potions. You also start with one of the following potions: \\
                \begin{tabular}{cccc}
                    \textbf{Felix Felisis} & \textbf{Polyjuice} & \textbf{Healing} & \textbf{Forgetfulness}\\
%                    \textbf{Griffindor} & \textbf{Hufflepuff} & \textbf{Ravenclaw} & \textbf{Slytherin}\\
                \end{tabular}
                \item[o] {\bf Transfiguration}: changing form of objects
            \end{itemize}

        \charactersection{Equipment}
        		What would a wizard be without \emph{stuff}? 
		You start with the following equipment, as well as one appropriate personal item of your choice.

        \begin{tabular}{l}
            \textbf{Item} \\
            Wand \\
            Set of core textbooks \\
            Set of student robes \\
            Quill and parchment set \\
        \end{tabular}

    \end{charactersheet}

% End document
\end{document}
